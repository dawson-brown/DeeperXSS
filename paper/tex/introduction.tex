Cross-site scripting (XSS) attacks persist as a major issue for web applications despite their root causes being well understood. The Open Web Application Security Project consistently ranks XSS in the top ten vulnerabilities on the web. For this reason, there exists a large body of work aimed at automating the detection and prevention of XSS attacks; this includes approaches using Machine Learning techniques to detect both attacks and vulnerabilities. DeepXSS is an LSTM classifier that is meant to detect XSS payloads; it was designed by Fang \textit{et. al} and is purported to have very high precision and recall rates. For our project, we intend to recreate and extend the work of DeepXSS by Fang \textit{et. al} \cite{fang2018deepxss}. We want to more clearly outline and comment on the strengths and weaknesses of their architecture as well as replicate their purported results.\cite{fang2018deepxss}. 

The primary motivation for this work is to verify Fang \textit{et. al}'s DeepXSS method of detecting cross-site scripting (XSS) attack payloads \cite{fang2018deepxss}. Given that XSS is a significant problem for many web applications, the need for further research into the detection of XSS attack payloads is apparent.  Unfortunately, we found that this paper lacked detail and failed to address key questions related to the work.  Given DeepXSS' promising results, it would be very useful to address the lack of detail, and attempt to replicate the methods used by the authors.  In verifying these results we could further our understanding of why DeepXSS was so effective (or why it was not as effective as it seemed) and apply lessons to future research.