\documentclass{llncs} 

\usepackage{algorithm}
\usepackage{algpseudocode}
\usepackage{seqsplit}
\usepackage{amsmath}
\newcommand{\url}[1]{{\ttfamily\seqsplit{#1}}}

\title{DeeperXSS: An Exploration of the DeepXSS Approach}
\author{Matthew Bush, matthew.bush@ryerson.ca\\Dawson Brown, dawson.brown@ryerson.ca}
\institute{Ryerson University}

\date{}
\begin{document}
\maketitle
\pagestyle{plain}

\begin{abstract}
DeeperXSS is a...

\keywords{First keyword  \and Second keyword \and Another keyword.}
\end{abstract}


\section{Introduction}
Cross-site scripting (XSS) attacks persist as a major issue for web applications despite their root causes being well understood. The Open Web Application Security Project consistently ranks XSS in the top ten vulnerabilities on the web. For this reason, there exists a large body of work aimed at automating the detection and prevention of XSS attacks; this includes approaches using Machine Learning techniques to detect both attacks and vulnerabilities. DeepXSS is an LSTM classifier that is meant to detect XSS payloads; it was designed by Fang \textit{et. al} and is purported to have very high precision and recall rates. For our project, we intend to recreate and extend the work of DeepXSS by Fang \textit{et. al} \cite{fang2018deepxss}. We want to more clearly outline and comment on the strengths and weaknesses of their architecture as well as replicate their purported results.\cite{fang2018deepxss}. 

The primary motivation for this work is to verify Fang \textit{et. al}’s DeepXSS method of detecting cross-site scripting (XSS) attack payloads \cite{fang2018deepxss}. Given that XSS is a significant problem for many web applications, the need for further research into the detection of XSS attack payloads is apparent.  Unfortunately, we found that this paper lacked detail and failed to address key questions related to the work.  Given DeepXSS’ promising results, it would be very useful to address the lack of detail, and attempt to replicate the methods used by the authors.  In verifying these results we could further our understanding of why DeepXSS was so effective (or why it was not as effective as it seemed) and apply lessons to future research.


\section{DeepXSS}
This section breifly outlines the DeepXSS approach and high-lights some its shortcomings as they relate to reproducibility.





\section{DeeperXSS: exploring DeepXSS}
In this section we outline our approach for recreating the DeepXSS architecture, including difficulties we had in reproducibility, limitations with DeepXSS that needed to be addressed, some creative liberties we took, and a few alternative machine learning architectures that proved interesting. 
\subsection{Data Preprocessing}
\subsubsection{Decoding}
We built a custom recursive URL decoder. This decoder performs a depth first search of 5 different decodings. At each level, the decoder tries to decode the URL with all decodings, for each decoding that is successfull, the decoder will recursively try to further decode the string that resulted from the decoding; see algorithm \ref{decoder} for a pseudocode implementation. The first string enoucountered that none of the decoders can decode is returned as the decoded string. The supported encodings are: URL unicode encoding (this includes characters of the form \%uxxxx and \textbackslash uxxxx),URL encoding (that is characters of the form \%xx), HTML character references (that is characters of the form \&xx;), hex encoding, and base64 encoding. 

The major steps of the algorithm are as follows: on line \ref{dec:all} all the decoder functions are called and passed the URL to be decoded. These functions all return a Tuple of the form (Boolean, String) where the Boolean value indicates if the decoding was successfull while the String is the decoded string (the String remains unchanged if the decoding fails). Next, on line \ref{dec:res}, the result tuples of all the attempted decoders are looped over. If it's found that a decoder was successful (line \ref{dec:check}) then recurisvely call the decoder on the resulting string (line \ref{dec:rec}). If the recursive decoding succeeds than return the result of the recursive call (\ref{dec:rec:true}). If the decoder gets to line \ref{dec:nrec}, than that means one of two things happened: either none of the decoders succeeded, in which case the string must be fully decoded and so we go to line \ref{dec:true}; if on the other hand some of the decoders succeeded, then if the algorithm gets to line \ref{dec:nrec} it must have been the case that none of the recursive calls succeeded (meaning line \ref{dec:rec:true} was never reached) which means the decoder was unable to decode the string, and line \ref{dec:false} is run. 

\begin{algorithm}
\caption{Recursive Decoder}\
\label{decoder}
\begin{algorithmic}[1]
    \Function{decode}{url}
        \State decoders $\gets$ [url(), unicode(), html(), hex(), base64()]
        \State dec\_results $\gets$ []
        \For{decoder \textbf{in} decoders} \label{dec:all}
            \State dec\_results\.append(decoder(url))
        \EndFor

        \State some\_decoded $\gets False$
        \For{decode\_success, decode\_str \textbf{in} dec\_results} \label{dec:res}
            \State some\_decoded $\gets$ decode\_success \textbf{or} some\_decoded
            \If{decode\_success} \label{dec:check}
                \State (next\_decode\_success, next\_decode\_str) = DECODE(decode\_str) \label{dec:rec}
                \If{next\_decode\_success}
                    \State \Return ($True$, next\_decode\_str) \label{dec:rec:true}
                \EndIf
            \EndIf 
        \EndFor

        \If{some\_decoded} \label{dec:nrec}
            \State \Return ($False$, url) \label{dec:false}
        \Else 
            \State \Return ($True$, url) \label{dec:true}
        \EndIf

    \EndFunction
\end{algorithmic}
\end{algorithm}

To clarify now what the decoder will actually do with a URL, consider the URL:

\url{http://example.com/706174682F746F2F66696C653F783D26616D703B6C743B73637269707426616D703B67743B253230616C65727428253230312532302925323026616D703B6C743B2F73637269707426616D703B67743B}. 

Passing this through the decoder, the Hex decoder will succeed and produce the string: 

\url{http://example.com/path/to/file?x=\&amp;lt;script\&amp;gt;\%20alert(\%201\%20)\%20\&amp;lt;/script\&amp;gt;} 

the the URL decoder will succeed giving the string: 

\url{http://example.com/path/to/file?x=\&amp;lt;script\&amp;gt; alert( 1 ) \&amp;lt;/script\&amp;gt;}

then the HTML decoder will succeed and give:

\url{http://example.com/path/to/file?x=\&lt;script\&gt; alert( 1 ) \&lt;/script\&gt;} 

and finally the HTML decoder will succeed again and give: 

\url{http://example.com/path/to/file?x=<script> alert( 1 ) </script>}. 

At this point, no decoders will succeed and so the decoded string will be returned on line \ref{dec:true}.


\subsubsection{Tokenization}
In DeepXSS they defined and looked for six different kinds of token summarized in table \ref{tok:tab}.

\begin{table}
\begin{center}
\begin{tabular}{||c | c||} 
    \hline
    Classification & List \\ [0.5ex] 
    \hline\hline
    \textbf{Start Label} &  $<$script$>$, $<$body$>$, $<$img , etc... \\ 
    \hline
    \textbf{End Label} & $</$script$>$, $</$body$>$, etc... \\
    \hline
    \textbf{Windows Event} & onerror=, onload=, onblur=, oncut=, etc... \\
    \hline
    \textbf{Function Name} & alert(, String.fromCharCode(, etc... \\
    \hline
    \textbf{Script URL} & javascript:, vbscript:, etc... \\ 
    \hline
    \textbf{Other} & $>$, ), \#, etc... \\ [1ex] 
    \hline
\end{tabular}
\caption{\label{tok:tab}DeepXSS Tokens.}
\end{center}
\end{table}

We expanded on this set of tokens and ended up with a total of 14 token types. The reason we expanded on this token set is because many many URLs especially benign URLs) contained zero tokens. For example \url{http://www.wittebeer.be/?oid=911\&pid=8056} or \url{http://www.facebook.com/Euphnet?sk=wall}, both of which are in the DMOZ directory, do not contain any of their token types. As such, we expanded their table to include 8 more as shown in table \ref{exp:tok:tab}.

\begin{table}
\begin{center}
\begin{tabular}{||c | c||} 
    \hline
    Classification & List \\ [0.5ex] 
    \hline\hline
    \textbf{Integer Argument} &  (543), (1), (2004), etc... \\ 
    \hline
    \textbf{Integer Constant} &  1, 2, 5432, 54 , etc... \\ 
    \hline
    \textbf{String Argument} & (``Hello''), (String.fromCharCode(65)), etc... \\
    \hline
    \textbf{Assignment LHS} & x=, variable=, etc... \\
    \hline
    \textbf{Assignment RHS} & =x, =654, =value, etc... \\
    \hline
    \textbf{Path} & path/ t56543-trer-yt43/, etc... \\ 
    \hline
    \textbf{Identifier} & iden, value, hello, goodbye, etc... \\ [1ex] 
    \hline
\end{tabular}
\caption{\label{exp:tok:tab}Expanded DeepXSS Tokens.}
\end{center}
\end{table}

\subsubsection{Generalization}
In keeping with DeepXSS \cite{fang2018deepxss}, we generalized many parts of the URL. 



\subsection{Word2Vec}



\subsection{LSTM Classifier}



\subsection{Evaluation}


\section{Comparison}
\begin{table}
\begin{center}
\begingroup
\setlength{\tabcolsep}{10pt} % Default value: 6pt
\renewcommand{\arraystretch}{1.5} % Default value: 1
\begin{tabular}{|| c | c | c ||} 
    \hline
    Prediction & XSS & Not XSS \\ 
    \hline\hline
    \textbf{Predicted XSS} &  $t_p$ & $f_p$ \\ 
    \hline
    \textbf{Predicted Not XSS} & $f_n$ & $t_n$ \\
    \hline
\end{tabular}
\endgroup
\caption{\label{conf-mat}Confusion Matrix}
\end{center}
\end{table}


\begin{align*}
    &\text{Precision} = \frac{t_p}{t_p + f_p} \\[10pt]
    &\text{Recall} = \frac{t_p}{t_p + f_n} \\[10pt]
    &\text{F1} = 2 \cdot \frac{\text{Precision} \cdot \text{Recall}}{\text{Precision} + \text{Recall}}\\[10pt]
    &\text{Accuracy} = \frac{t_p + t_n}{t_p + t_n + f_p + f_n}
\end{align*}

\begin{table}
\begin{center}
\begingroup
\setlength{\tabcolsep}{10pt} % Default value: 6pt
\renewcommand{\arraystretch}{1.5} % Default value: 1
\begin{tabular}{|| c | c | c | c | c ||} 
    \hline
    Model & Precision & Recall & F1 & Accuracy \\ 
    \hline\hline
    \textbf{DeepXSS} &  0.995 & 0.979 & 0.987 & n/a \\ 
    \hline
    \textbf{DeeperXSS:softmax} & 0.989 & 0.973 & 0.981 & 0.981 \\
    \hline
    \textbf{DeeperXSS:sigmoid} & 0.988 & 0.976 & 0.982 & 0.983 \\
    \hline
    \textbf{DeeperXSS:sequence} & 0.991 & 0.956 & 0.973 & 0.975 \\
    \hline
    \textbf{DeeperXSS:random} & 0.099 & 0.097 & 0.098 & 0.56 \\
    \hline
\end{tabular}
\endgroup
\caption{\label{comparison}Model Comparison.}
\end{center}
\end{table}


\section{Related Work}
Mokbal \textit{et al.} created a multilayer perceptron (MLP) model for detecting XSS both in dynamic webpages and URLs \cite{mokbal2019mlpxss}. Their approach, called MLPXSS, has three main pillars: data scraping, feature extraction, and an artificial neural network (ANN). Their feature extraction level has three modules to extract HTML-based features, Javascript-based features, and URL-based features. The HTML module tokenizes tags, attributes and events--focusing on things that trigger Javascript execution (like \textit{href} or \textit{onclick}). The Javascript module parses and tokenizes Javascript code that is pulled from a webpage creating an abstract syntax tree. There are various ways to include Javascript in a page like script tags, \textit{onclick} and \textit{onsubmit} calls, \textit{href}, etc... Lastly they tokenize potentially malicious parts of URLs, like HTML properties, tags, some keywords (\textit{login}, \textit{signup}), \textit{document} references, and various special characters like `$<$', `$>$' and `$/$'. The MLP is trained on token streams with a sigmoid output layer. Their perceptron had precision, f-measure, and accuracy all in excess of 99\% \cite{mokbal2019mlpxss}.

Zhang \textit{et al.} propose a dual Gaussian mixture model (GMM) approach that trains two seperate GMMs models (one for benign and one for malicious) and then combines their outputs to make a prediction. Additionally, they train the models on both the URL and the server response to the URL in an attempt to get richer features \cite{zhang2019cross}. To preprocess the URLs, they decode, tokenize, and train a Word2Vec model to retrieve a vector representation of each token. Their tokenization approach is very similar to that of DeepXSS and MPLXSS, they however inclue the domain and path for the benign GMM. They are however generalized to simply `domain' and `path' \cite{zhang2019cross}. Their reasoning for this is that containing just a domain and path is characteristic of a benign URL, whereas an XSS URL is characterized by its maliciously constructed parameters and not the presence of a domain and path \cite{zhang2019cross}. Their models can be trained on requests, responses, or both. They reason that in many cases, benign requests contain no XSS tokens, which isn't very interesting, however responses contain useful features for both XSS and none-XSS tokenization. Their multi-stage dual GMM using both responses and requests greatly improved classification \cite{zhang2019cross}.

Goswami \textit{et al.} propose an attribute clustering technique to perform unsupervised grouping of malicious and benign scripts. They're feature extraction is wholly different from DeepXSS and other deep learning classifiers. They propose 15 features that characterize malicious and benign scripts creating a 16-dimensional vector for each script (including class) \cite{goswami2017unsupervised}. These features are meta-features like length of the script, number of strings and the average string length, number of methods, number of unicode and hex characters, among others. These features are then min-max normalized before clustering \cite{goswami2017unsupervised}. Their algorithm was able to achieve an accuracy in excess of 98\% \cite{goswami2017unsupervised}.  

The authors in \cite{afzal2021deeplearning} focus on classifying broadly defined malicious URLs sent through email or over social networks.  In this case a malicious URL is any URL that could result in harm to the user visiting it.  They propose a hybrid deep-learning approach called \textit{URLdeepDetect} to extract semantic features from URLs to classify them as either benign or malicious.  The preprocessing stages tokenizes various parts of the URL before applying word-level embedding through Word2Vec.  The embedded tokens are then fed to an LSTM where samples are classified based on LSTM outputs or k-means clustering.  This paper claims 98.3\% accuracy for LSTM classification of malicious URLs and 99.7\% accuracy with k-means clustering.  The authors claim that the success of their approach is in part due to the Word2Vec token embedding and maintaining URL sequence as it provides the model with more semantic information for classification.

The work done in \cite{vishnu2014prediction} explores XSS detection using three machine learning algorithms: Naïve Bayes, Support Vector Machine, and J48 Decision Tree.  This paper attempts to detect reflected XSS, persistent XSS, and DOM based XSS meaning it requires web page scripts as well as URLs.  The malicious URLs and scripts were collected from the XXSed \cite{xssed} project and the benign samples were collected from the Dmoz open directory project \cite{dmoz}.  This work, however, only uses five features for scripts (e.g., number of characters, request for cookie, etc.) and seven features for URLs (e.g., number of characters, presence of script tags, etc.) rather than extracting tokens to represent the entire URL or script.  Each model was then subjected to 10-fold cross validation and the results were compared.  J48 performed the best based on features from both URL and JavaScript achieving a 99\% true positive rate and 99\% precision.  The authors also found that discretized attributes provided the best classifier results for J48.

Detecting XSS attacks on social networking sites is the primary focus of \cite{rathore2017xss}.  The authors propose an approach consisting of feature identification, web page collection, feature extraction, building a training dataset, and using a machine learning algorithm to classify web pages as XSS or non-XSS.  This approach extracts features from URLs, HTML tags, and the host social networking site.  The features of these web pages are very coarse-grained, consisting of things like the maximum size of URLs, and counts of harmful keywords.  The authors then train ten classifiers with a RandomForest classifier achieving 97.7\% precision and 97.1\% recall.  They suggest future work to enhance the feature set and apply more machine learning algorithms such as deep learning.


\section{Discussion and Concludion}
\input{conc}


\newpage
\bibliography{ref}
\bibliographystyle{splncs04}

\end{document}